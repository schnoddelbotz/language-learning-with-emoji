\subsection{zählen - counting - compter - contare}
% 🇩🇪🇬🇧🇫🇷🇮🇹
\emoji{shrug} 🤷‍♂️ L'\leolink{fr}{ordinateur} is used to manage your money at the bank counter. Man kann darauf zählen, dass der Computer das Konto im Griff hat.

\begin{tabular}{ |c|l|l|l|l| }
  \hline
  \emoji{grinning-face}  & \emoji{de}           & \emoji{uk}     & \emoji{fr}         & \emoji{it} \\
  \hline
  % \vocnum{}{}{}{}{}
  \vocnum{0 }{null      }{zero     }{zéro    }{zero       }
  \vocnum{1 }{eins      }{one      }{un      }{uno        }
  \vocnum{2 }{zwei      }{two      }{deux    }{due        }
  \vocnum{3 }{drei      }{three    }{trois   }{tre        }
  \vocnum{4 }{vier      }{four     }{quatre  }{quattro    }
  \vocnum{5 }{fünf      }{five     }{cinq    }{cinque     }
  \vocnum{6 }{sechs     }{six      }{six     }{sei        }
  \vocnum{7 }{sieben    }{seven    }{sept    }{sette      }
  \vocnum{8 }{acht      }{eight    }{huit    }{otto       }
  \vocnum{9 }{neun      }{nine     }{neuf    }{nove       }
  \vocnum{10}{zehn      }{ten      }{dix     }{dieci      }
  \vocnum{11}{elf       }{eleven   }{onze    }{undici     }
  \vocnum{12}{zwölf     }{twelve   }{douze   }{dodici     }
  \vocnum{13}{dreizehn  }{thriteen }{treize  }{tredici    }
  \vocnum{14}{vierzehn  }{fourteen }{quatorze}{quattordici}
  \vocnum{15}{fünfzehn  }{fifteen  }{quinze  }{quindici   }
  \vocnum{16}{sechszehn }{sixteen  }{seize   }{sedici     }
  \vocnum{17}{siebzehn  }{seventeen}{dix-sept}{diciassette}
  \vocnum{18}{achtzehn  }{eighteen }{dix-huit}{diciotto   }
  \vocnum{19}{neunzehn  }{nineteen }{dix-neuf}{diciannove }
  \vocnum{20}{zwanzig   }{twenty   }{vingt   }{venti      }
  % \vocnum{}{}{}{}{}
  \vocnum{30}{dreißig   }{thirty   }{trente  }{trenta     }
  \vocnum{40}{vierzig   }{fourty   }{quarante}{quaranta   }
  \hline
\end{tabular}

Counting from 1 to 20 in French, using macOS' say command: \href{https://raw.githubusercontent.com/schnoddelbotz/language-learning-with-emoji/main/french-1-20.mp3}{Listen on github.com}

\emoji{clown-face}
If your brain runs PHP, try \url{https://github.com/patrickschur/number-to-words/blob/master/src/NumberToWords/Locale/German.php}. Sollten Sie andersherum ticken probieren Sie doch einmal \url{https://pypi.org/project/word2number-i18n/}, um ausgeschriebene Zahlen (wie einundzwanzig) ins
\href{https://de.wikipedia.org/wiki/Dezimalsystem}{Dezimalsystem} zu überführen (21) - bei Lust und Laune.
You may guess how much IT people love converting between words and numbers - because why would you even...? Internationally. For JavaScript, try \url{https://github.com/yamadapc/js-written-number/}.


\subsection{ordnen - sort/order - ordonner - ordinare}
%  🤷‍♂️ Ich bestelle - hier meine Order! Bringt das in Ordnung oder es donnert!

From \url{https://en.wikipedia.org/wiki/Ordinal_numeral}:
\begin{quote}
In linguistics, ordinal numerals or ordinal number words are words representing position or rank in a sequential order; the order may be of size, importance, chronology, and so on (e.g., "third", "tertiary"). They differ from cardinal numerals, which represent quantity (e.g., "three") and other types of numerals. [...]
Ordinal numbers may be written in English with numerals and letter suffixes: 1st, 2nd or 2d, 3rd or 3d, 4th, 11th, 21st, 101st, 477th, etc., with the suffix acting as an ordinal indicator.
\end{quote}
